\section{Development}

\subsection{REST server}

\paragraph{}{
    A REST-let server is a software that handle HTTP requests to exchange data.
 Mainly four statements of the HTTP protocol is : \texttt{GET}, \texttt{POST},
 \texttt{PUT} and \texttt{DELETE}. REST doesn't specify how to use it, so below 
 I detail how I implement my REST server and specify the different requests 
 handled. To exchange the data, I use JSON documents.
}

\paragraph{\texttt{GET} requests and replies}{
    To get the list of topics, the messages of a topic and a message, I use 
 \texttt{GET} request as describe below.
}

\begin{tabular}{|c|c|p{4.3cm}|p{3.4cm}|}
\hline
\# & Request & keyword & action \\
\hline
1 & \texttt{GET} & \texttt{TOPICS} & get list of topics names \\
\hline
2 & \texttt{GET} & \texttt{MESSAGES} \& \texttt{TOPIC\_NAME} & get list of message about a topic \\
\hline
3 & \texttt{GET} & \texttt{MESSAGES\_WITH\_TIMESTAMPS} \& \texttt{TOPIC\_NAME} & get list of messages with their time-stamps \\
\hline
4 & \texttt{GET} & \texttt{MESSAGE} \& \texttt{MESAGE\_ID} & get a message \\
\hline
\end{tabular}

\paragraph{}{
    Replies are JSON documents containing the requested data. If there is no
 data suitable (or if there is an error) on the server, empty documents are 
 send. The table below describe the return JSON values regarding the request.
}


\begin{tabular}{|c|c|p{2cm}|p{2cm}|p{3cm}|}
\hline
\# & JSON type & JSON keys & type & description \\
\hline
1 & list & & \texttt{string} & name of topics \\
\hline
2 & list & & \texttt{string} & list of messages IDs about a topic \\
\hline
3 & collection & \texttt{key} \& \texttt{value} & \texttt{string} \& \texttt{long} & map of messages : message ID $\rightarrow$ time-stamp, about a topic \\
\hline
4 & object & \texttt{ID}, \texttt{SENDER}, \texttt{TOPIC}, \texttt{TIMESTAMP}, \texttt{CONTENT} & \texttt{TIMESTAMP} is a \texttt{long}, \texttt{string} for all & a message \\
\hline
\end{tabular}

\paragraph{\texttt{POST} request and reply}{
    To send a message to the server, I use the \texttt{POST} method. The object
 send to the server is the same as the one received when a client asks for a
 message. The query send by the client contains two values : \texttt{POST} and 
 \texttt{MESSAGE} which is the message in a JSON document.
}

\paragraph{}{
    I did not implement the messages attachments but the REST server and the
 client object (to implement the my protocol) are totally functional. To run the
 REST server simply run the class \texttt{RunServerREST} class from the jar file
 produce by the ant build. A client can be easily implemented by using the 
 \texttt{RestClient} class included in the jar file. An example of such a client
 is also include in the jar file named \texttt{TestClientRest}.
}

\subsection{SOAP service}

\paragraph{}{
    A SOAP service is basically an \texttt{aar} file loaded by the Apache Axis 
 server. Such a service can be development by generating the code of the server 
 and the client from a \texttt{WSDL} file or by generating the \texttt{WSDL}
 from java classes representing the messages exchange between a SOAP client and
 a SOAP server. I decided, as shown during the tutorial, to generate the Java 
 code from a \texttt{WSDL file}. To do so, I specified the types I used and the
 messages exchanged for each requests. The full \texttt{WSDL} file can be found 
 at appendix \ref{app:wsdl}.
}

\paragraph{}{
    Unfortunately, I spend a lot time trying to fix an issue I got, the Axis 
 server load the service but was unable to find the right classes despite the
 fact they all are in the \texttt{aar} artifact. Finally, I fixed the problem
 by changing the packages name. But I still have an error coming from my server
 and the way I create the different request. So, for now, my service runs but it
 can't be used since I have to fix and figure out how I can create \texttt{XML}
 replies with the types I specify in the \texttt{WSDL} file.
}

